% Options for packages loaded elsewhere
\PassOptionsToPackage{unicode}{hyperref}
\PassOptionsToPackage{hyphens}{url}
%
\documentclass[
]{article}
\usepackage{amsmath,amssymb}
\usepackage{iftex}
\ifPDFTeX
  \usepackage[T1]{fontenc}
  \usepackage[utf8]{inputenc}
  \usepackage{textcomp} % provide euro and other symbols
\else % if luatex or xetex
  \usepackage{unicode-math} % this also loads fontspec
  \defaultfontfeatures{Scale=MatchLowercase}
  \defaultfontfeatures[\rmfamily]{Ligatures=TeX,Scale=1}
\fi
\usepackage{lmodern}
\ifPDFTeX\else
  % xetex/luatex font selection
\fi
% Use upquote if available, for straight quotes in verbatim environments
\IfFileExists{upquote.sty}{\usepackage{upquote}}{}
\IfFileExists{microtype.sty}{% use microtype if available
  \usepackage[]{microtype}
  \UseMicrotypeSet[protrusion]{basicmath} % disable protrusion for tt fonts
}{}
\makeatletter
\@ifundefined{KOMAClassName}{% if non-KOMA class
  \IfFileExists{parskip.sty}{%
    \usepackage{parskip}
  }{% else
    \setlength{\parindent}{0pt}
    \setlength{\parskip}{6pt plus 2pt minus 1pt}}
}{% if KOMA class
  \KOMAoptions{parskip=half}}
\makeatother
\usepackage{xcolor}
\usepackage[margin=1in]{geometry}
\usepackage{color}
\usepackage{fancyvrb}
\newcommand{\VerbBar}{|}
\newcommand{\VERB}{\Verb[commandchars=\\\{\}]}
\DefineVerbatimEnvironment{Highlighting}{Verbatim}{commandchars=\\\{\}}
% Add ',fontsize=\small' for more characters per line
\usepackage{framed}
\definecolor{shadecolor}{RGB}{248,248,248}
\newenvironment{Shaded}{\begin{snugshade}}{\end{snugshade}}
\newcommand{\AlertTok}[1]{\textcolor[rgb]{0.94,0.16,0.16}{#1}}
\newcommand{\AnnotationTok}[1]{\textcolor[rgb]{0.56,0.35,0.01}{\textbf{\textit{#1}}}}
\newcommand{\AttributeTok}[1]{\textcolor[rgb]{0.13,0.29,0.53}{#1}}
\newcommand{\BaseNTok}[1]{\textcolor[rgb]{0.00,0.00,0.81}{#1}}
\newcommand{\BuiltInTok}[1]{#1}
\newcommand{\CharTok}[1]{\textcolor[rgb]{0.31,0.60,0.02}{#1}}
\newcommand{\CommentTok}[1]{\textcolor[rgb]{0.56,0.35,0.01}{\textit{#1}}}
\newcommand{\CommentVarTok}[1]{\textcolor[rgb]{0.56,0.35,0.01}{\textbf{\textit{#1}}}}
\newcommand{\ConstantTok}[1]{\textcolor[rgb]{0.56,0.35,0.01}{#1}}
\newcommand{\ControlFlowTok}[1]{\textcolor[rgb]{0.13,0.29,0.53}{\textbf{#1}}}
\newcommand{\DataTypeTok}[1]{\textcolor[rgb]{0.13,0.29,0.53}{#1}}
\newcommand{\DecValTok}[1]{\textcolor[rgb]{0.00,0.00,0.81}{#1}}
\newcommand{\DocumentationTok}[1]{\textcolor[rgb]{0.56,0.35,0.01}{\textbf{\textit{#1}}}}
\newcommand{\ErrorTok}[1]{\textcolor[rgb]{0.64,0.00,0.00}{\textbf{#1}}}
\newcommand{\ExtensionTok}[1]{#1}
\newcommand{\FloatTok}[1]{\textcolor[rgb]{0.00,0.00,0.81}{#1}}
\newcommand{\FunctionTok}[1]{\textcolor[rgb]{0.13,0.29,0.53}{\textbf{#1}}}
\newcommand{\ImportTok}[1]{#1}
\newcommand{\InformationTok}[1]{\textcolor[rgb]{0.56,0.35,0.01}{\textbf{\textit{#1}}}}
\newcommand{\KeywordTok}[1]{\textcolor[rgb]{0.13,0.29,0.53}{\textbf{#1}}}
\newcommand{\NormalTok}[1]{#1}
\newcommand{\OperatorTok}[1]{\textcolor[rgb]{0.81,0.36,0.00}{\textbf{#1}}}
\newcommand{\OtherTok}[1]{\textcolor[rgb]{0.56,0.35,0.01}{#1}}
\newcommand{\PreprocessorTok}[1]{\textcolor[rgb]{0.56,0.35,0.01}{\textit{#1}}}
\newcommand{\RegionMarkerTok}[1]{#1}
\newcommand{\SpecialCharTok}[1]{\textcolor[rgb]{0.81,0.36,0.00}{\textbf{#1}}}
\newcommand{\SpecialStringTok}[1]{\textcolor[rgb]{0.31,0.60,0.02}{#1}}
\newcommand{\StringTok}[1]{\textcolor[rgb]{0.31,0.60,0.02}{#1}}
\newcommand{\VariableTok}[1]{\textcolor[rgb]{0.00,0.00,0.00}{#1}}
\newcommand{\VerbatimStringTok}[1]{\textcolor[rgb]{0.31,0.60,0.02}{#1}}
\newcommand{\WarningTok}[1]{\textcolor[rgb]{0.56,0.35,0.01}{\textbf{\textit{#1}}}}
\usepackage{graphicx}
\makeatletter
\def\maxwidth{\ifdim\Gin@nat@width>\linewidth\linewidth\else\Gin@nat@width\fi}
\def\maxheight{\ifdim\Gin@nat@height>\textheight\textheight\else\Gin@nat@height\fi}
\makeatother
% Scale images if necessary, so that they will not overflow the page
% margins by default, and it is still possible to overwrite the defaults
% using explicit options in \includegraphics[width, height, ...]{}
\setkeys{Gin}{width=\maxwidth,height=\maxheight,keepaspectratio}
% Set default figure placement to htbp
\makeatletter
\def\fps@figure{htbp}
\makeatother
\setlength{\emergencystretch}{3em} % prevent overfull lines
\providecommand{\tightlist}{%
  \setlength{\itemsep}{0pt}\setlength{\parskip}{0pt}}
\setcounter{secnumdepth}{-\maxdimen} % remove section numbering
\ifLuaTeX
  \usepackage{selnolig}  % disable illegal ligatures
\fi
\usepackage{bookmark}
\IfFileExists{xurl.sty}{\usepackage{xurl}}{} % add URL line breaks if available
\urlstyle{same}
\hypersetup{
  pdftitle={IQ\_Script},
  pdfauthor={Douglas Venegas},
  hidelinks,
  pdfcreator={LaTeX via pandoc}}

\title{IQ\_Script}
\author{Douglas Venegas}
\date{2024-10-28}

\begin{document}
\maketitle

{
\setcounter{tocdepth}{3}
\tableofcontents
}
Run the script to line 910 for an index based on your annotation library
or run the script to line 64 and jump to line 910 for IQ based on
NPAtlas annotations.

\section{Packages instalation}\label{packages-instalation}

Load or install pacman to streamline package management

\begin{Shaded}
\begin{Highlighting}[]
\ControlFlowTok{if}\NormalTok{ (}\SpecialCharTok{!}\FunctionTok{require}\NormalTok{(}\StringTok{"pacman"}\NormalTok{)) }\FunctionTok{install.packages}\NormalTok{(}\StringTok{"pacman"}\NormalTok{)}
\NormalTok{pacman}\SpecialCharTok{::}\FunctionTok{p\_load}\NormalTok{(}\StringTok{"RColorBrewer"}\NormalTok{, }\StringTok{"tidyverse"}\NormalTok{, }\StringTok{"readxl"}\NormalTok{, }\StringTok{"rvest"}\NormalTok{, }\StringTok{"dplyr"}\NormalTok{, }\StringTok{"tidyr"}\NormalTok{, }\StringTok{"igraph"}\NormalTok{, }\StringTok{"visNetwork"}\NormalTok{, }
               \StringTok{"readr"}\NormalTok{, }\StringTok{"ggplot2"}\NormalTok{, }\StringTok{"ggraph"}\NormalTok{, }\StringTok{"graphTweets"}\NormalTok{, }\StringTok{"writexl"}\NormalTok{, }\StringTok{"rJava"}\NormalTok{, }\StringTok{"rcdk"}\NormalTok{, }\StringTok{"proxy"}\NormalTok{, }\StringTok{"httr"}\NormalTok{)}
\end{Highlighting}
\end{Shaded}

\begin{verbatim}
## Error in download.file(url, destfile, method, mode = "wb", ...) : 
##   download from 'http://cran.rstudio.com/bin/windows/contrib/4.4/rJava_1.0-11.zip' failed
\end{verbatim}

\begin{verbatim}
## Error in download.file(url, destfile, method, mode = "wb", ...) : 
##   download from 'http://cran.rstudio.com/bin/windows/contrib/4.4/rcdk_3.8.1.zip' failed
\end{verbatim}

\begin{Shaded}
\begin{Highlighting}[]
\CommentTok{\# Set JAVA\_HOME environment variable for rJava}
\FunctionTok{Sys.setenv}\NormalTok{(}\AttributeTok{JAVA\_HOME =} \StringTok{\textquotesingle{}C:/Program Files/Java/jdk{-}17\textquotesingle{}}\NormalTok{)}
\end{Highlighting}
\end{Shaded}

\section{Inputs}\label{inputs}

Load network from a .graphML file. Change the file name as necessary

\begin{Shaded}
\begin{Highlighting}[]
\NormalTok{graphml }\OtherTok{\textless{}{-}} \FunctionTok{read\_graph}\NormalTok{(}\StringTok{"Data/network\_Crusemann.graphml"}\NormalTok{, }\AttributeTok{format =} \StringTok{"graphml"}\NormalTok{)}
\NormalTok{Metadata }\OtherTok{\textless{}{-}} \FunctionTok{read\_delim}\NormalTok{(}\StringTok{"Data/Metadata\_Cruseman.csv"}\NormalTok{, }
                       \AttributeTok{delim =} \StringTok{";"}\NormalTok{, }\AttributeTok{escape\_double =} \ConstantTok{FALSE}\NormalTok{, }\AttributeTok{trim\_ws =} \ConstantTok{TRUE}\NormalTok{)}
\end{Highlighting}
\end{Shaded}

\section{List of directories to
create}\label{list-of-directories-to-create}

\begin{Shaded}
\begin{Highlighting}[]
\NormalTok{Dirs }\OtherTok{\textless{}{-}} \FunctionTok{c}\NormalTok{(}\StringTok{"Data/CSV\_IQ"}\NormalTok{, }\StringTok{"Data/IQ\_Plot"}\NormalTok{,}\StringTok{"Data/CSV\_IQNPA"}\NormalTok{, }\StringTok{"Data/IQNPA\_Plot"}\NormalTok{)}
\ControlFlowTok{for}\NormalTok{ (dir }\ControlFlowTok{in}\NormalTok{ Dirs) \{}
  \ControlFlowTok{if}\NormalTok{ (}\SpecialCharTok{!}\FunctionTok{dir.exists}\NormalTok{(dir)) \{}
    \FunctionTok{dir.create}\NormalTok{(dir, }\AttributeTok{recursive =} \ConstantTok{TRUE}\NormalTok{)}
\NormalTok{  \}}
\NormalTok{\}}
\end{Highlighting}
\end{Shaded}

\section{Get node and edge data from the graph and prepare
data}\label{get-node-and-edge-data-from-the-graph-and-prepare-data}

\begin{Shaded}
\begin{Highlighting}[]
\ControlFlowTok{if}\NormalTok{ (}\FunctionTok{is\_igraph}\NormalTok{(graphml)) \{}
\NormalTok{  df\_edges }\OtherTok{\textless{}{-}} \FunctionTok{as\_data\_frame}\NormalTok{(graphml, }\AttributeTok{what =} \StringTok{"edges"}\NormalTok{)}
\NormalTok{  df\_nodes }\OtherTok{\textless{}{-}} \FunctionTok{as\_data\_frame}\NormalTok{(graphml, }\AttributeTok{what =} \StringTok{"vertices"}\NormalTok{)}
\NormalTok{\}}

\DocumentationTok{\#\# Calculate total nodes per sample}
\NormalTok{Total\_nodes\_per\_sample }\OtherTok{\textless{}{-}}\NormalTok{ df\_nodes }\SpecialCharTok{\%\textgreater{}\%}
  \FunctionTok{select}\NormalTok{(id, UniqueFileSources, MQScore) }\SpecialCharTok{\%\textgreater{}\%} 
  \FunctionTok{separate\_rows}\NormalTok{(UniqueFileSources, }\AttributeTok{sep =} \StringTok{"}\SpecialCharTok{\textbackslash{}\textbackslash{}}\StringTok{|"}\NormalTok{) }\SpecialCharTok{\%\textgreater{}\%}  
  \FunctionTok{group\_by}\NormalTok{(UniqueFileSources)}

\CommentTok{\# Merge with metadata to get strain information}
\NormalTok{Total\_nodes\_per\_strain }\OtherTok{\textless{}{-}} \FunctionTok{merge}\NormalTok{(Total\_nodes\_per\_sample, Metadata, }
                                \AttributeTok{by.x =} \StringTok{"UniqueFileSources"}\NormalTok{, }\AttributeTok{by.y =} \StringTok{"filename"}\NormalTok{, }\AttributeTok{all.x =} \ConstantTok{TRUE}\NormalTok{)}

\CommentTok{\# Select relevant columns for analysis}
\NormalTok{Total\_nodes\_per\_strain }\OtherTok{\textless{}{-}}\NormalTok{ Total\_nodes\_per\_strain }\SpecialCharTok{\%\textgreater{}\%}
  \FunctionTok{select}\NormalTok{(id, ATTRIBUTE\_strain, MQScore)}

\CommentTok{\# Count total nodes per strain}
\NormalTok{Total\_nodes }\OtherTok{\textless{}{-}} \FunctionTok{table}\NormalTok{(Total\_nodes\_per\_strain}\SpecialCharTok{$}\NormalTok{ATTRIBUTE\_strain)}
\NormalTok{Total\_nodes }\OtherTok{\textless{}{-}} \FunctionTok{as.data.frame}\NormalTok{(Total\_nodes)}
\FunctionTok{colnames}\NormalTok{(Total\_nodes) }\OtherTok{\textless{}{-}} \FunctionTok{c}\NormalTok{(}\StringTok{"ATTRIBUTE\_strain"}\NormalTok{, }\StringTok{"Count"}\NormalTok{)}
\end{Highlighting}
\end{Shaded}


\end{document}
